\section{Results}
\label{sec:results}

\subsection{Single simulation runs}

% 1. Representative plots, and narrative, from single simulations

\subsection{Simulation suites and parameter sweeps}

% (All with suite size 1200)
% Regardless of voting makeup:
%    - * Single-chaser chase dist .28 all the way to .05 at 8th election
%        Multi-chasers chase dist .28 only down to .15 at 8th election
%    This will be plot 3a and 3b
% Default voting makeup:
%    - Rationality: doesn't matter how many chasers. Rationality sucks.
%    - * Winning: one chaser gets incredible benefit, if there's only one.
%    This will be plot 4a and 4b
% No rational voters:
%    - Rationality: doesn't matter how many chasers. Rationality sucks.
%    - * Winning: chaser doesn't get any appreciable benefit.
%    This will be plot 5
% No FF voters:
%    - Rationality: doesn't matter how many chasers. Rationality overall better.
%    - * Winning: one chaser gets incredible benefit, if there's only one.
% All party voters:
%    - Rationality: Rationality REALLY sucks. Each additional chaser makes this
%    noticeably worse (though not stat sig).
%    - * Winning: chaser doesn't get any appreciable benefit.
% All rational voters:
%    - Rationality: Rationality uniformly perfect.
%    - * Winning: one chaser gets incredible benefit, if there's only one.
%      * Now if there are two, they both get incredible benefit.
%    This will be plot 6a and 6b  (contrast with 4a and 4b)
% All FF voters:
%    - Rationality: doesn't matter how many chasers. Rationality really sucks.
%    - * Winning: chaser doesn't get any appreciable benefit.

% Question for us to figure out: why, if default voter alg, only one chaser
% gets any benefit (two chasers cancel each others' benefits out stat sig) but
% if all voters are rational, then two chasers both benefit.


% Some preliminary findings and questions:

% chasing does pay off if lots of rational voters.
% chasing does not pay off if lots of party voters.
% Is there a correlation between how much each candidate chased (total sum of Euclidean distances of all chase actions) and how often they won

% does chasing ever hurt a candidate? hypothesis: if voters easily switch
% parties, and a candidate "overchases" and outruns their base, they will lose
% their base and this will hurt them. Can we reproduce this?

% Is there a correlation between how much the voting population is drifting and how likely the election at that time is to be rational?





\section{Results}
\label{sec:results}

\subsection{Parameter settings}

In all of the following results, we use a suite size of 1200 (that is, 1200
independent simulations with different random number seeds for \textit{each}
combination of distinct independent variable values) and run each simulation
for 400 iterations. We set $V_N=20$ voters, $C_N=3$ candidates, $I_N=3$ issues,
$p_e=.2$ edge probability, and $E=50$ for eight consecutive elections in the
400 iteration time period.

\subsection{Multiple chasers impede one another}

Using the default voting algorithm distribution ($\frac{1}{3}$ rational,
$\frac{1}{3}$ party, $\frac{1}{6}$ F\&F1, $\frac{1}{6}$ F\&F2)
When only one of three candidates is chasing votes, it reaches its ``sweet
spot'' early on, and quickly encounters diminishing returns to further chasing.
When two candidates are chasing, they take longer to reach their equilibrium.
(See Figure~\ref{one_vs_two_chasers}.) We speculate that this is because the
chasers are interfering with one another as they court voters in opinion space,
dampening the gains of their rival by moving into regions that might have been
unspoken for.

\begin{figure}[ht]
\centering
\includegraphics[width=0.45\textwidth]{assets/one_chaser_maxes_out_soon.png}
\includegraphics[width=0.45\textwidth]{assets/two_chasers_max_out_later.png}
\caption{A single chasing candidate maxes out its gains more quickly than two
competing chasers do.}
\label{one_vs_two_chasers}
\end{figure}

This effect can be further illustrated by looking at how many elections are
actually won by chasing vs. non-chasing candidates.
Figure~\ref{chasing_winners} shows the proportion of elections won by each
candidate in a three-candidate race with one chaser (left side) and with two
(right side). (Error bars are 95\% confidence intervals for a proportion,
assuming normality.) As you can see, when only one candidate chases voters, it
has a tremendous advantage over the other candidates, and this advantage
increases the longer that the drifting/chasing mechanism continues. Two
chasers, however, interfere with one another such that each gets only a modest
benefit.

\begin{figure}[ht]
\centering
\includegraphics[width=0.45\textwidth]{assets/one_chaser_big_benefit.png}
\includegraphics[width=0.45\textwidth]{assets/two_chasers_small_benefit.png}
\caption{Using the default voting algorithm distribution, a single chasing
candidate (red) has an enormous advantage over its competitors. If two
candidates (both red and blue) are chasing voters, however, neither gains
nearly as much benefit. (Note: the small advantages that red consistently has
over blue are an artifact of how the simulation handles ties -- it awards
victory to the lowest-numbered tied candidate. This is miniscule in comparison
with the main effect illustrated here, however.)}
\label{chasing_winners}
\end{figure}

% No rational voters:
%    - Rationality: doesn't matter how many chasers. Rationality sucks.
%    - * Winning: chaser doesn't get any appreciable benefit.
% No FF voters:
%    - Rationality: doesn't matter how many chasers. Rationality overall better.
%    - * Winning: one chaser gets incredible benefit, if there's only one.
% All party voters:
%    - Rationality: Rationality REALLY sucks. Each additional chaser makes this
%    noticeably worse (though not stat sig). Also, over time it gets noticeably
%    - Winning: chaser doesn't get any appreciable benefit. (sanity check, duh)
%    This will be plot 5: all_party_1_chaser_no_benefit.png
% All rational voters:
%    - Rationality: Rationality uniformly perfect.
%    - * Winning: one chaser gets incredible benefit, if there's only one.
%      * Now if there are two, they both get incredible benefit.
%    This will be plot 6a and 6b  (contrast with 4a and 4b):
%       all_rat_one_chaser_even_bigger_benefit.png,
%         all_rat_two_chasers_even_bigger_benefit.png

% (HP re-ran these: check them in email.)
% All FF voters:
%    - Rationality: doesn't matter how many chasers. Rationality really sucks.
%    - * Winning: chaser doesn't get any appreciable benefit.

% Question for us to figure out: why, if default voter alg, only one chaser
% gets any benefit (two chasers cancel each others' benefits out stat sig) but
% if all voters are rational, then two chasers both benefit.


% Some preliminary findings and questions:

% chasing does pay off if lots of rational voters.
% chasing does not pay off if lots of party voters.
% Is there a correlation between how much each candidate chased (total sum of Euclidean distances of all chase actions) and how often they won

% does chasing ever hurt a candidate? hypothesis: if voters easily switch
% parties, and a candidate "overchases" and outruns their base, they will lose
% their base and this will hurt them. Can we reproduce this?

% Is there a correlation between how much the voting population is drifting and how likely the election at that time is to be rational?





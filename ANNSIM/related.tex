\section{Related work}
\label{sec:related}

Recently, Gao \textit{et al}\cite{gao_forecasting_2022} have presented a
promising approach to election forecasting. They combine current demographic
attributes of the population with historical results of past election outcomes
in order to determine how much each attribute is likely to influence voters,
and in which direction. This is very different from our model, both in purpose
(actually forecasting upcoming elections, rather than exploring ``what if?''
voting strategy scenarios) and in approach (the agents in
\cite{gao_forecasting_2022} do not influence one another on a social network.)

The emotion/information/opinion (E/I/O) approach, utilized by both
\cite{sobkowicz_quantitative_2016} and, more recently,
\cite{burke_quantitatively_2022}, can lead to less rigidity in model outcomes
and perhaps even conditions conducive to legitimate third parties coming into
play.

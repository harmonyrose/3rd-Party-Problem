\section{Related work}
\label{sec:related}

% Mention: our work is actually more similar with Axelrod's cultural influence
% model than most of the election modeling stuff

%\subsection{Political opinions and homophily}
%
%The opinion dynamics of individuals on political (or other) matters is a rich
%field; see Lorenz\cite{lorenz_continuous_2007} and
%Mastroeni\cite{mastroeni_agent-based_2019} for surveys. A dominant factor in
%all such studies is the principle of homophily\cite{mcpherson_birds_2001}:
%people tend to prefer to interact with others who are similar to them in some
%way (\textit{e.g.}, with respect to the opinions they hold), and become more
%similar to them over time.

\subsection{Factors that influence voting}

To model a political system, the nuanced psychological factors that influence
voting decisions need to be meaningfully consolidated down to only the most
significant components. It is a complex field of study; prior to even choosing
a candidate, citizens must first decide whether or not they will even turn out,
which entails its own distinct psychological process. To narrow the scope of
our study, we assume all agents will vote, allowing us to focus on the dynamics
and minds of politically active citizens.

Redlawsk and Habegger\cite{redlawsk_citizens_2020} consolidated the findings
of political psychologists to coin several ideal-type voters that illustrate
how different strategies of information processing manifest in voters'
decisions. The driving mechanism that dictates how citizens make voting
decisions can be described by the natural trade-off that exists between making
a good decision and an easy decision. We explore the effect voters have on a
political system when they sacrifice a full assessment of the information
environment for a quick, uncomplicated decision. 


\subsection{Modeling voter opinions}

In many approaches to modeling opinion dynamics, each agent's opinion comprises
merely a valence and possibly a magnitude, rather than involving a more complex
amalgamation of issue positions. The original binary voter model, for example,
utilized only a single ``issue'' upon which an individual could hold a
position.\cite{holley_ergodic_1975,clifford_model_1973}. Even in Kottonau and
Paul-Wohstl's much more recent seminal work\cite{kottonau_simulating_2004},
citizens maintain a ``mental accounting'' of only a single variable over the
course of an election season -- namely, which of the two parties they favor.
The campaigns for these parties then attempt to optimize their advertising
resources to influence would-be voters at opportune times. Burke and
Searle\cite{burke_quantitatively_2022} followed Sobkowicz's
emotion/information/opinion (E/I/O) approach\cite{sobkowicz_quantitative_2016}
in making agents' mental states more complex and realistic: in addition to
holding a certain opinion about a party (or candidate), an agent is, at any
moment, in either a ``calm'' or ``agitated'' state, which controls how they
influence and are influenced by other agents (and by campaign messaging). Here,
too, the object of analysis is a single ``issue'' (pro party A or B), not an
array of them.

Various researchers have probed the idea of voters possessing an array of issue
positions (or other
attributes)\cite{axelrod_dissemination_1997,fortunato_vector_2005,weisbuch_meet_2002,schweighofer_agent-based_2020,jung_cultural_2021,sirbu_opinion_2013, meyer_importance_2023}
but we know of few on applying these ideas specifically to elections.

\subsection{Modeling the election process}

One election-centric study that did employ agents with an array of opinions is
the older but seminal work by Kollman \textit{et
al}\cite{kollman_adaptive_1992}. This work explored similar themes to ours, but
with discrete-valued opinions, no variation in voting algorithms among voters,
and exactly two parties (incumbent and challenger) whose strategies encompassed
an entire sequence of consecutive elections. Our multi-party model, by
contrast, features a heterogeneous electorate in which different voters use
different decision procedures for candidate selection. Parties have visibility
into voter issue positions (an analogy to opinion polls) and can adjust their
party's platform in between elections in order to woo them, but face the wild
card of an unknown proportion of agents not voting rationally. The interplay
between these campaign choices and the diverse ways in which voters actually
choose how to vote is a novel contribution of this paper.

Other existing agent-based election models focus on the continuous dynamics of
political systems in which all voters have fixed opinions and are assumed to
always support the candidate closest to them in the opinion space
\cite{kollman_adaptive_1992,laver_policy_2005,lehrer_governator_2018,
wright_modeling_2015}. In his formative work, Laver\cite{laver_policy_2005}
studies how candidates can use different issue adjustment campaign strategies
to gain more support. Lehrer and Schumacher\cite{lehrer_governator_2018} build
upon Laver's foundation by modeling party formations and coalition governments,
and Wright and Sengupta\cite{wright_modeling_2015} study the impact of
oligarchs on a democratic model.

Recently, Gao \textit{et al}\cite{gao_forecasting_2022} have presented a
promising approach to election forecasting. They combine current demographic
attributes of the population with historical results of past election outcomes
in order to determine how much each attribute is likely to influence voters,
and in which direction. This is very different from our model, both in purpose
(actually forecasting upcoming elections, rather than exploring ``what if?''
voting strategy scenarios) and in approach (the agents in
\cite{gao_forecasting_2022} do not influence one another on a social network.)


% Harmony's unorganized notes
%Laver:
%\begin{itemize}
%\item analyzes different issue adjustment campaign strategies party leaders can use to get %more support from voters. our model uses the same issue adjustmnet strategy across the board
%\item does not hold elections in which voters actually vote; it is assumed that voters support the leader with the closest opinions to them
%\item voters do not interact or change their opinions
%\item deployed model in a way to retrieve a real opinion poll series in Ireland
%\end{itemize}
%Lehrer et al:
%\begin{itemize}
%\item extension of Laver's model
%\item  parties can form coalition governments
%\item  parties have utility functions that determine how they use issue adjustment to %campaign; they are motivated by office pay-offs as well as votes
%\item they look at how different party adaption strategies can lead to opitmal voter %representation in both the legislature and the government, whereas we look at how %different voting algorithms can lead to optimal voter representation in the legislature
%\item they hold elections, but all voters vote rationally whereas we explore different %voting algorithms
%\item voters do not interact or change opinions
%\end{itemize}
%Wright and Sengupta:
%\begin{itemize}
%political system
%\item voters do not interact or change opinions, and they all vote rationally at every step
%\item parties campaign by issue adjustment, considering an additional "olig-salience" %variable that voters possess that represents how important the special interests of the %oligarchs are to them in comparison to the importance of policy positions
%\end{itemize}

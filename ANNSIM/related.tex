\section{Related work}
\label{sec:related}

% Key aspects of SD/HP: graph, opinion space, elections, modeling (not
% forecasting)

% No opinion space
In many approaches to modeling elections, voters' attitudes towards the parties
comprises merely a valence and a magnitude, rather than involving a more
complex amalgamation of issue positions. For example, in Kottonau and
Paul-Wohstl's seminal work\cite{kottonau_simulating_2004}, citizens maintain
a ``mental accounting'' of the parties over the course of an election, and the
campaigns attempt to optimize their advertising resources to influence would-be
voters at opportune times. In our model, by contrast, the main levers available
to a campaign are the party's issue positions, which they can adjust in an
attempt to woo voters.

% Election forecasting
Recently, Gao \textit{et al}\cite{gao_forecasting_2022} have presented a
promising approach to election forecasting. They combine current demographic
attributes of the population with historical results of past election outcomes
in order to determine how much each attribute is likely to influence voters,
and in which direction. This is very different from our model, both in purpose
(actually forecasting upcoming elections, rather than exploring ``what if?''
voting strategy scenarios) and in approach (the agents in
\cite{gao_forecasting_2022} do not influence one another on a social network.)

The emotion/information/opinion (E/I/O) approach, utilized by both
\cite{sobkowicz_quantitative_2016} and, more recently,
\cite{burke_quantitatively_2022}, can lead to less rigidity in model outcomes
and perhaps even conditions conducive to legitimate third parties coming into
play.

% Harmony's unorganized notes
Laver:
\begin{itemize}
\item looks at the continuous dynamics of political competition with no regard for elections
\item analyzes different strategies party leaders can continuously employ to get support from voters- they treat this as an IV, we treat it as a DV
\item deployed model in a way to retrieve a real opinion poll series in Ireland
\end{itemize}
Lehrer et al:
\begin{itemize}
\item extension of Laver's model: parties form coalition governments and parties have utility functions (they are motivated by more than just votes)
\item they look at how different party adaption strategies can lead to opitmal representation in both parties and the government, whereas we look at how different voting algorithms lead to different vote distributions
\item they hold elections, but all voters vote rationally whereas we explore different voting algorithms
\end{itemize}
Wright and Sengupta:
\begin{itemize}
\item extension of Laver's model: uses their "hunter" campaigning  strategy
\item analyzes the impact of lobbying oligarchs on the political system
\item no elections or voting, they only look at party dynamics
\end{itemize}

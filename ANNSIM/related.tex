\section{Related work}
\label{sec:related}

\subsection{Political opinions and homophily}
    1. Homophily as a near-universal driver of human behavior **(SD)**

\subsection{Factors that influence voting}
%    2. Factors that go into voting **(HP)**
%        * choice of candidate
%        * whether to "turn out"
Understanding the psychological factors underlying a citizen’s decision to vote is integral in modeling a political system. Though complex and nuanced, Redlawsk and Habegger [cite] consolidated the findings of political psychologists to coin several ideal-type voters that illustrate how different strategies of information processing manifest in voters’ decisions. The driving mechanism that dictates how citizens make voting decisions is the natural tradeoff that exists between making a good decision and an easy decision. We explore the effect voters have on a political system when they sacrifice a full assessment of the information environment for a quick, uncomplicated decision. Prior to choosing a candidate, citizens must also decide whether or not they will turn out, which entails another distinct psychological process. To narrow the scope of our study, we assume all agents will vote, allowing us to focus on the dynamics of politically active citizens, rather than the general populace.

\subsection{Opinion Dynamics models}
    1. Lots of OD lit, but remarkably little on continuous opinion vectors **(SD)**

\subsubsection{Continuous opinion vectors}
    2. Other OD approaches to continuous opinion vectors **(SD)**

\subsubsection{Modeling the elecction process}
    3. Approaches to actually modeling the election process **(HP)**
        * note that this is different than election forecasting


% Key aspects of SD/HP: graph, opinion space, elections, modeling (not
% forecasting)

% Mention: our work is actually more similar with Axelrod's cultural influence
% model than most of the election modeling stuff

% No opinion space
In many approaches to modeling elections, voters' attitudes towards the parties
comprises merely a valence and possibly a magnitude, rather than involving a
more complex amalgamation of issue positions. The original binary voter model,
for example, utilized only a single ``issue'' upon which an individual could
hold a position.\cite{holley_ergodic_1975,clifford_model_1973}. Even in
Kottonau and Paul-Wohstl's much more recent seminal
work\cite{kottonau_simulating_2004}, citizens maintain a ``mental accounting''
of only a single variable over the course of an election season -- namely,
which of the two parties they favor. The campaigns for these parties then
attempt to optimize their advertising resources to influence would-be voters at
opportune times. Burke and Searle\cite{burke_quantitatively_2022} followed
Sobkowicz's emotion/information/opinion (E/I/O)
approach\cite{sobkowicz_quantitative_2016} in making agents' mental states more
complex and realistic: in addition to holding a certain opinion about a party
(or candidate), an agent is, at any moment, in either a ``calm'' or
``agitated'' state, which controls how they influence and are influenced by
other agents (and by campaign messaging). Here, too, the object of analysis is
a single ``issue'' (pro party A or B), not an array of them.

Our model, in contrast to all of these, the main levers available to a campaign
are the party's issue positions, which they can adjust in an attempt to woo
voters.


% Opinion space done differently

% Various researchers have probed the idea of a vector of opinions (or other
% attributes), such as Axelrod 1997, Fortunato 2005, Weisbuch 2002, and those
% mentioned in Lorenz's 2007 and Mastroeni et al's 2019 surveys. None of these
% involve elections or candidates, though. Weisbuch and some others use a
% simplex, which is a different appproach entirely.

% Weisbuch
% Jung et al 2021
%   multiple issues, but opinions on each one is binary
%   adjacent-numbered issues are presumed to be semantically related to each
%     other. an agent having all 0's, or all 1's, is interpreted as being
%     "consistent" in some way.
%   the main influencing dynamic is not pairwise, but *groupwise*. (considering
%     an agent's entire ingroup.)

% Sîrbu et al 2013  "vector of values, but not in the same way as ours;
%   instead, the vector represents multiple possible positions on a single
%   issue (e.g., agent A is .3 "permanent cease fire in Gaza", .2 "temporary
%   cease fire in Gaza," .5 "increase US funding to Israel in fighting
%   Hamas.")"




% Election forecasting
Recently, Gao \textit{et al}\cite{gao_forecasting_2022} have presented a
promising approach to election forecasting. They combine current demographic
attributes of the population with historical results of past election outcomes
in order to determine how much each attribute is likely to influence voters,
and in which direction. This is very different from our model, both in purpose
(actually forecasting upcoming elections, rather than exploring ``what if?''
voting strategy scenarios) and in approach (the agents in
\cite{gao_forecasting_2022} do not influence one another on a social network.)

The emotion/information/opinion (E/I/O) approach, utilized by both
\cite{sobkowicz_quantitative_2016} and, more recently,
\cite{burke_quantitatively_2022}, can lead to less rigidity in model outcomes
and perhaps even conditions conducive to legitimate third parties coming into
play.

% Harmony's unorganized notes
Laver:
\begin{itemize}
\item analyzes different issue adjustment campaign strategies party leaders can use to get more support from voters. our model uses the same issue adjustmnet strategy across the board
\item does not hold elections in which voters actually vote; it is assumed that voters support the leader with the closest opinions to them
\item voters do not interact or change their opinions
\item deployed model in a way to retrieve a real opinion poll series in Ireland
\end{itemize}
Lehrer et al:
\begin{itemize}
\item extension of Laver's model
\item  parties can form coalition governments
\item  parties have utility functions that determine how they use issue adjustment to campaign; they are motivated by office pay-offs as well as votes
\item they look at how different party adaption strategies can lead to opitmal voter representation in both the legislature and the government, whereas we look at how different voting algorithms can lead to optimal voter representation in the legislature
\item they hold elections, but all voters vote rationally whereas we explore different voting algorithms
\item voters do not interact or change opinions
\end{itemize}
Wright and Sengupta:
\begin{itemize}
\item extension of Laver's model
\item analyzes the impact of oligarchs (that can make significant donations) on the political system
\item voters do not interact or change opinions, and they all vote rationally at every step
\item parties campaign by issue adjustment, considering an additional "olig-salience" variable that voters possess that represents how important the special interests of the oligarchs are to them in comparison to the importance of policy positions
\end{itemize}

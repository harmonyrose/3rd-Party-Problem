\section{Experimentation}
\label{sec:experimentation}

\subsection{Verification}

To confirm the model's basic operations, we ran it interactively with a wide
variety of values for every parameter. We verified, among other things, that
the CI2 mechanism does asymptote towards a fixed point as expected.
Figure~\ref{drifts_and_chases_asymptote} (left) depicts the average amount that each
agent's opinion was either ``pushed'' (or ``pulled'') towards an agent similar
enough (or dissimilar enough) on the comparison issue. These drift distances
decay over the course of about four hundred iterations until an equilibrium is
reached. Empirically, we discovered that an \textbf{openness} parameter value
of \textbf{.1}, and a \textbf{pushaway} of \textbf{.6}, worked well for
producing the CI2 effect.

\begin{figure}[ht]
\centering
\includegraphics[width=0.45\textwidth]{assets/drifts_asymptote.png}
\includegraphics[width=0.45\textwidth]{assets/chase_dists_asymptote.png}
\caption{Both voters' drift movements (due to the CI2 mechanism approaching steady
state) and candidates' chase movements (due to all candidates reaching local
maxima in state space) asymptote to zero.}
\label{drifts_and_chases_asymptote}
\end{figure}

Similarly, as expected, the distances in opinion space that candidates
traversed in ``chasing'' voters also trend downward to zero. After reaping the
nearby ``low-hanging fruit'' (voters not locked in by any other candidate and
thus prime for poaching), candidates face diminishing returns when chasing any
further would jeopardize their existing voters. See
Figure~\ref{drifts_and_chases_asymptote} (right) for an illustration with two
chasing candidates.


\subsection{Independent variables}

%- i.v.'s: SD focuses on candidate actions (i.e., how much they chase, etc.) and
%  HP focuses on voter actions (ie., the mix of algorithms)
%
%SD holds fixed: 1/3rd, 1/3rd, 1/6th, 1/6th
%    party switch thresh .2
%    openness .1
%    pushaway .6
%    edge probability .2
%
%HP holds fixed:
%    chase radius .2
%    num_chasers 3
%    num_candidates 3
%
%
%Both hold fixed:
%    num_opinions
%    N
%    
%- d.v.'s: one of us focuses on candidate winning, and the other focuses on
%  election rationality  


\subsection{Dependent variables}

% **(SD)**
% D.v.’s
%   Likelihood that election #n will turn out rational, for 1 <= n <= 8.
%   Candidate winning
%   Party switches

\section{Experimentation}
\label{sec:experimentation}

\subsection{Verification}

To confirm the model's basic operations, we ran it interactively with a wide
variety of values for every parameter. We verified, among other things, that
the CI2 mechanism does asymptote towards a fixed point as expected.
Figure~\ref{drifts_and_chases_asymptote} (left) depicts the average amount that
each agent's opinion was either ``pushed'' (or ``pulled'') towards an agent
similar enough (or dissimilar enough) on the comparison issue. These drift
distances decay over the course of about four hundred iterations until an
equilibrium is reached. Empirically, we discovered that an \textbf{openness}
parameter value of \textbf{.1}, and a \textbf{pushaway} of \textbf{.6}, worked
well for producing the CI2 effect.

\begin{figure}[ht]
\centering
\includegraphics[width=0.45\textwidth]{assets/drifts_asymptote.png}
\includegraphics[width=0.45\textwidth]{assets/chase_dists_asymptote.png}
\caption{Both voters' drift movements (due to the CI2 mechanism approaching steady
state) and candidates' chase movements (due to all candidates reaching local
maxima in state space) asymptote to zero.}
\label{drifts_and_chases_asymptote}
\end{figure}

Similarly, as expected, the distances in opinion space that candidates
traversed in ``chasing'' voters also trend downward to zero. After reaping the
nearby ``low-hanging fruit'' (voters not locked in by any other candidate and
thus prime for poaching), candidates face diminishing returns when chasing any
further would jeopardize their existing voters. See
Figure~\ref{drifts_and_chases_asymptote} (right) for an illustration with two
chasing candidates. Note that the voters approach zero mean drift distance more
quickly than the candidates approach zero mean chase distance. This is to be
expected since candidates are (1) only chasing each election (every $E$
iterations) instead of every iteration, (2) self-imposing a limit (the chase
radius) on how aggressively they will pursue voters, and (3) reacting to
changes in voter opinions only after voters have made them.


\subsection{Independent variables}

The important independent variables we focus on in this paper center around the
action choices of the two kinds of agents. Candidates can be either chasers or
non-chasers, and can alter how far they are willing to move in opinion space in
the pursuit of additional voters. Voters adopt different voting algorithms, and
thus the electorate can be divided up into groups of varying sizes: for
example, all rational; half rational and half party; half party and one-fourth
of each of the two Fast and Frugal variants; and so on.

% If changing ER edge prob is interesting, add that here.

\subsection{Dependent variables}

The main dependent variables we examine are the rationality of election
outcomes (how often the winner is the same candidate who would have been
elected if all voters had voted rationally) and, in the case of heterogeneous
candidates (some chasers and some non-chasers, and/or chasers with different
chase radii) how often they win elections at different stages of the
simulation.

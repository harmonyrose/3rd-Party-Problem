\documentclass{scspaperproc}

\usepackage{latexsym}
\usepackage{graphicx}
\usepackage{mathptmx}
\usepackage{ulem}

% To do: here's how Stephen compiles the paper from the ANNSIM directory:
% pdflatex scs23paper.tex 
% bibtex scs23paper
% pdflatex scs23paper.tex
% pdflatex scs23paper.tex
%
% By putting this into an alias in my .bashrc with:
% alias lt="pdflatex scs23paper.tex ; bibtex scs23paper ; pdflatex % scs23paper.tex ; pdflatex scs23paper.tex"
% I can compile the paper to a PDF in one shot.
%
% When adding new papers to Zotero, be sure to put them in the ANNSIM folder
% (or subfolder), then highlight all and Export... to BibTex mode, saving the
% file under the name bib.bib (in assets subdirectory). Don't forget to commit
% your changes to bib.bib as well as your changes to any of the .text files!


\usepackage[pdftex,colorlinks=true,urlcolor=blue,citecolor=black,anchorcolor=black,linkcolor=black,bookmarks=false]{hyperref}

% Avoid overrunning the right margin; you are welcome to remove this, provided that you take care not to overrun the right margin anywhere in your paper
\sloppy

\begin{document}

\SCSpagesetup{Davies and Peura}

\def\SCSconferencename{Annual Simulation Conference}

\def\SCSconferenceacro{ANNSIM'24}

\def\SCSpublicationyear{2024}

\def\SCSconferenceeditors{P.J. Giabbanelli, I. David, C. Ruiz-Martin, B. Oakes and R. C\'{a}rdenas}

\def\SCSconferencedates{May 20-23}

\def\SCSconferencevenue{American University, DC, USA}

\title{THE INTERACTION BETWEEN HETEROGENEOUS VOTING STRATEGIES AND DYNAMIC
VOTE-SEEKING CAMPAIGNS: AN AGENT-BASED MODEL}

\textit{(Author names/affiliations to be redacted for initial blind submission.)}
\author[\authorrefmark{1}]{\sout{Stephen Davies}}
\author[\authorrefmark{1}]{\sout{Harmony Peura}}

\affil[\authorrefmark{1}]{\sout{University of Mary Washington Computer Science,
Virginia USA}}
\affil[ ]{\textit {\sout{stephen@umw.edu}}}
\affil[ ]{\textit {\sout{hpeura@umw.edu}}}


\maketitle

\section*{Abstract}

Political candidates in a democracy articulate positions on the issues of the
day, but they are also highly aware of voter sentiment on those issues, and
tailor their campaigns accordingly as they seek to win elections. Voters, too,
adjust their political opinions based on (among other things) interactions with
others in their social network. We present an agent-based simulation that
models this dynamic interplay between candidates and voters, in order to shed
light on what outcomes candidates can expect to result from a policy of
``chasing'' votes. The voters in our simulation differ from one another in the
decision procedure they use in choosing who to vote for -- these voting
algorithms are modeled on results from the political science literature about
the different ways voters make decisions. Our model can thus be used to
experiment with a virtual electorate, to determine the conditions under which
vote-chasing candidates gain an advantage or perhaps even cause the election
outcome to be objectively irrational.


\textbf{Keywords:} opinion dynamics, ABM, election, voting

\section{Introduction}
\label{sec:intro}

Many factors are known to influence voters in democratic elections, including
their own background characteristics (demographics, ideology, partisanship,
personality traits, \textit{etc.}), the distinctiveness of the candidates, the
availability of information, and the voter's perception of their
identity.\cite[pp.3-4]{redlawsk_citizens_2020} But another major factor -- and
one we would argue \textit{should} be the most important, if citizens are
casting their votes rationally -- is the candidates' stances on the issues of
the day. At least in principle, voters are electing the candidate who will
enact policies most favorable to them if elected to office.

Voters are known to adjust their positions on issues over time, in response to
influence from their
peers\cite{dandekar_biased_2013,mcpherson_birds_2001,mas_differentiation_2013}
among other sources. They are also known to prefer interactions with those who
are similar to them in some respect (the well-known effect of
``homophily''\cite{boucher_structural_2015,centola_homophily_2007}) and prone
to sever social ties with those who believe
differently\cite{skoric_what_2018,paik_defriending_2023}.

Further, political parties and candidates are known to adjust their positions
on issues in a strategic attempt to appeal to greater numbers of
voters.\cite{worcester_voter_2006} This phenomenon, termed ``policy
positioning''\cite{mcelroy_policy_2012} or ``issue
adaptation''\cite{benoit_issue_2002} means that, like voters, candidates
dynamically vary their articulated political opinions over time. Neither voters
nor candidates occupy a fixed position in opinion space.

This dynamic interplay between voters influencing one another and candidates
seeking to gain their allegiance is worth studying, in part because the results
it will produce are unknown at the outset. Will candidates who ``chase'' voters
in this way succeed in winning more elections? Or will they alienate their base
by deserting policy positions they previously held, wiping out these gains? If
voters' opinions are ``drifting'' in response to one another between election
cycles, will candidates be chasing a moving target and thus outfox themselves?

We purport to answer such questions with an agent-based model that reproduces
this dance in opinion space between voters and candidates. Opinions on issues
are represented as vectors of values in a continuous n-dimensional space (as in
works like \cite{fortunato_vector_2005,lorenz_continuous_2007}), and voters
interact with one another on a randomly-generated social network, influencing
each other on these issues. Concurrently, candidates evaluate their estimated
vote totals and adjust their policy positions in an attempt to maximize them.
Voters are each equipped with one of a variety of empirically-verified voting
strategies, and cast their ballots according to that algorithm. With this
model, we can study the effects of different voting algorithms within the
electorate, and different strategic choices on the part of candidates, on
election results.



\section{Related work}
\label{sec:related}

Recent work by \cite{gao_forecasting_2022} indicates that election forecasting
can be improved by using an agent-based model in addition to, or in place of,
classical forecasting metrics such as surveys, polls, and fundamentals.

The emotion/information/opinion (E/I/O) approach, utilized by both
\cite{sobkowicz_quantitative_2016} and, more recently,
\cite{burke_quantitatively_2022}, can lead to less rigidity in model outcomes
and perhaps even conditions conducive to legitimate third parties coming into
play.

\section{The model}
\label{sec:model}

\section{Experimentation}
\label{sec:experimentation}

% From Jan 3 meeting:
% I.v.’s
%   Pushaway and openness thresholds (set to .1 and .6)
%   Number of parties (set to 3)
%   Threshold for voters changing parties (set to same as chase dist)
%   Chase dist: how far candidates let themselves chase (set to same as party switch thresh)
%   Fractions of the voting algorithms (sweep)
% 
% 
% D.v.’s
%   Likelihood that election #n will turn out rational, for 1 <= n <= 8.
%   Time to stability (both for voter opinions (possibly measured by # of buckets) and for candidate opinions)

\section{Results}
\label{sec:results}

\subsection{Parameter settings}

In all of the following results, we use a suite size of 1200 (that is, 1200
independent simulations with different random number seeds for \textit{each}
combination of distinct independent variable values) and run each simulation
for 400 iterations. We set $V_N=20$ voters, $C_N=3$ candidates, $I_N=3$ issues,
$p_e=.2$ edge probability, and $E=50$ for eight consecutive elections in the
400 iteration time period.

\subsection{Multiple chasers impede one another}

Using the default voting algorithm distribution ($\frac{1}{3}$ rational,
$\frac{1}{3}$ party, $\frac{1}{6}$ F\&F1, $\frac{1}{6}$ F\&F2)
When only one of three candidates is chasing votes, it reaches its ``sweet
spot'' early on, and quickly encounters diminishing returns to further chasing.
When two candidates are chasing, they take longer to reach their equilibrium.
(See Figure~\ref{one_vs_two_chasers}.) We speculate that this is because the
chasers are interfering with one another as they court voters in opinion space,
dampening the gains of their rival by moving into regions that might have been
unspoken for.

\begin{figure}[ht]
\centering
\includegraphics[width=0.45\textwidth]{assets/one_chaser_maxes_out_soon.png}
\includegraphics[width=0.45\textwidth]{assets/two_chasers_max_out_later.png}
\caption{A single chasing candidate maxes out its gains more quickly than two
competing chasers do.}
\label{one_vs_two_chasers}
\end{figure}

% Default voting makeup:
%    - Rationality: doesn't matter how many chasers. Rationality sucks.
%    - * Winning: one chaser gets incredible benefit, if there's only one.
%    This will be plot 4a and 4b
% No rational voters:
%    - Rationality: doesn't matter how many chasers. Rationality sucks.
%    - * Winning: chaser doesn't get any appreciable benefit.
%    This will be plot 5
% No FF voters:
%    - Rationality: doesn't matter how many chasers. Rationality overall better.
%    - * Winning: one chaser gets incredible benefit, if there's only one.
% All party voters:
%    - Rationality: Rationality REALLY sucks. Each additional chaser makes this
%    noticeably worse (though not stat sig).
%    - * Winning: chaser doesn't get any appreciable benefit.
% All rational voters:
%    - Rationality: Rationality uniformly perfect.
%    - * Winning: one chaser gets incredible benefit, if there's only one.
%      * Now if there are two, they both get incredible benefit.
%    This will be plot 6a and 6b  (contrast with 4a and 4b)
% All FF voters:
%    - Rationality: doesn't matter how many chasers. Rationality really sucks.
%    - * Winning: chaser doesn't get any appreciable benefit.

% Question for us to figure out: why, if default voter alg, only one chaser
% gets any benefit (two chasers cancel each others' benefits out stat sig) but
% if all voters are rational, then two chasers both benefit.


% Some preliminary findings and questions:

% chasing does pay off if lots of rational voters.
% chasing does not pay off if lots of party voters.
% Is there a correlation between how much each candidate chased (total sum of Euclidean distances of all chase actions) and how often they won

% does chasing ever hurt a candidate? hypothesis: if voters easily switch
% parties, and a candidate "overchases" and outruns their base, they will lose
% their base and this will hurt them. Can we reproduce this?

% Is there a correlation between how much the voting population is drifting and how likely the election at that time is to be rational?





\section{Discussion}
\label{sec:discussion}

\subsection{Candidate chasing}

%1. General findings about candidate "chasing." Do candidates tend to chase the
%same amount? Do high-chasers tend to do better in elections than low-chasers?
%Etc. **(SD)**

\subsection{Electorate's voting algorithm composition}

%2. General findings about voter algorithm mix. How much does this matter? How
%often are elections rational based on various compositions? Do certain voter
%algorithms make drifting less/more effective? **(HP)**

\section{Future work}
\label{sec:future}

% HP: pick 2 or three and discuss briefly

More than, and less than, 3 candidates

candidates chasing more often than just once per election (based on polling
data between elections)

the "constrained rational" voting alg

different social network types and parameters

issue importance (related to F\&F)

a non-fixed voting population (voters enter/exit, influenced by parents)

\subsection{``Micro-chasing''}
%2. If campaigns can microtarget which voters are likely to use which voting
%algorithms, can they use that to their advantage in drifting more
%strategically? **(SD)**

\input{conclusions.tex}


\bibliographystyle{scsproc}

\bibliography{assets/bib}



\section*{Author Biographies}

\textit{(To be redacted for initial blind submission.)}

\sout{\textbf{\uppercase{Stephen Davies}} is an international spy, sought-after
public speaker, and former member of the Jedi Council on Coruscant. He's
considered armed and dangerous and should be approached with caution.
His email address is \email{stephen@umw.edu}.}

\sout{\textbf{\uppercase{Harmony Peura}} is an Honors student in the University of
Mary Washington Computer Science department, and also sings alto for several
acapella groups. She has climbed Mt.~Everest on three separate occasions. Her
email address is \email{hpeura@umw.edu}.}


\end{document}

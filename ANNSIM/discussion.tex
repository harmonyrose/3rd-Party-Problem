\section{Discussion}
\label{sec:discussion}

% this is all HP's brain dump from her sticky note

\subsection{Candidate chasing}

%1. General findings about candidate "chasing." Do candidates tend to chase the
%same amount? Do high-chasers tend to do better in elections than low-chasers?
%Etc. **(SD)**

\subsection{Electorate's voting algorithm composition}

%2. General findings about voter algorithm mix. How much does this matter? How
%often are elections rational based on various compositions? Do certain voter
%algorithms make drifting less/more effective? **(HP)**

Both the rationality of elections and the efficacy of candidate chasing depend
crucially on the makeup of the electorate. We find that a balanced distribution
of rational, party-line, and fast \& frugal voters is conducive to benefiting
one chasing candidate. If two candidates are chasing votes, we observe a much
more modest advantage. As we see in the real-world with politicians, adopting
centrist-leaning opinions to appeal to the masses is a sound strategy. However, 
this strategy has a diminishing return on investment as two candidates fight over
the same voters in a "race to the center," providing an unexpected advantage to
a third candidate who has a solidified voter base.

When the electorate is comprised of party-line or fast \& frugal voters and
candidates are chasing votes, they engage in a blind dance of sorts, with
neither candidates nor voters knowing what is best for them. We might have
expected this dynamic to have little to no effect; sometimes, by chance,
it will work in everyone's favor, and sometimes it won't. What we actually
see though is a decrease in the rationality of the system as the dynamic
unfolds. The blind dance becomes a race to the bottom in which candidates
are fruitlessly chasing votes and voters are unknowingly electing candidates
that don't represent their beliefs. This effect is a bit more modest with
fast \& frugal voters than it is with party-line, but the trend remains
in any voter distribution that is not rational.

Chasing votes through opinion adjustment proves to be an extremely beneficial
campaigning strategy when the electorate is comprised of rational voters. In
this environment, one or two chasers stands to gain a significant number of
votes by chasing, proving this algorithm is effective in an ideal system.

These results may explain why real campaigns rely on strategies like emotional
messaging and playing up hot topic issues. Despite the fact that matching
voters where they are opinion-wise is an optimal strategy to \textit{represent} 
the greatest number of voters, the electorate is not rational. Oftentimes, they
will not elect the candidate that represents their beliefs the most. Rather,
they will make the easy, and "good enough" decision to vote with their party
identity or vote based on one or two important issues. 

So, while chasing is an exceptional strategy for a rational, or even partially
rational electorate, there comes a point where if enough voters are making
trade-offs for a quick decision, chasing only serves to make matters worse. 
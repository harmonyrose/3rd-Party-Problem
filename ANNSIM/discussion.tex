\section{Discussion}
\label{sec:discussion}

Both the rationality of elections and the efficacy of candidate chasing depend
crucially on the makeup of the electorate. 
Chasing votes through opinion adjustment proves to be an extremely beneficial
campaigning strategy when the electorate is comprised of rational voters. In
this environment, any number of chasers stand to gain a significant number of
votes by chasing, proving this algorithm is effective in an ideal system.

We find that a balanced distribution of rational, party-line, and fast \&
frugal voters is conducive to benefiting one chasing candidate. If two
candidates are chasing votes, we observe a much more modest advantage. As we
see with real political campaigns, adopting centrist-leaning opinions to appeal
to the masses is a sound strategy. However, this strategy has a diminishing
return on investment as two candidates fight over the same voters who aren't
guaranteed to vote rationally, providing an unexpected advantage to a third
candidate who has a solidified voter base.

When the electorate is comprised entirely of party-line or fast \& frugal
voters and candidates are chasing votes, they end up engaging in a blind dance
of sorts, where neither candidates nor voters know what is best for them. We
might have expected this dynamic to have little to no effect on the rationality
of election outcomes. It seemed probable that by chance, it would sometimes
work in everyone's favor, and other times it wouldn't. However, we observe a
consistent \textit{decrease} in the rationality of elections as this dynamic
unfolds. The blind dance becomes a race to the bottom in which candidates are
fruitlessly chasing votes and voters are unknowingly electing candidates that
don't represent their beliefs. This effect is a bit more modest with fast \&
frugal voters than it is with party-line, but the trend remains in any voter
distribution that is not rational.

Chasing votes through opinion adjustment proves to be an extremely beneficial
campaigning strategy when the electorate is comprised of rational voters. In
this environment, one or two chasers stand to gain a significant number of
votes by chasing, proving this algorithm is effective in an ideal system.

These results may explain why real campaigns rely on strategies like emotional
messaging and playing up hot topic issues. Despite the fact that matching
voters where they are opinion-wise is an optimal strategy to \textit{represent}
the greatest number of voters, the electorate is not rational. Oftentimes, they
will not elect the candidate that most closely represents their beliefs.
Rather, they will make the easy, and ``good enough'' decision by single-issue
voting of blind party voting.

So, while chasing is an exceptional strategy for a rational, or even partially
rational electorate, there comes a point where if enough voters are making
trade-offs for a quick decision, chasing only serves to make matters worse. 

\pagebreak

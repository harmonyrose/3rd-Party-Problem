\section{Introduction}
\label{sec:intro}

Many factors are known to influence voters in democratic elections, including
their own background characteristics (demographics, ideology, partisanship,
personality traits, \textit{etc.}), the distinctiveness of the candidates, the
availability of information, and the voter's perception of their
identity.\cite[pp.3-4]{redlawsk_citizens_2020} But another major factor -- and
one we would argue \textit{should} be the most important, if citizens are
casting their votes rationally -- is the candidates' stances on the issues of
the day. At least in principle, voters are electing the candidate who will
enact policies most favorable to them if elected to office.

Voters are known to adjust their positions on issues over time, in response to
influence from their
peers\cite{dandekar_biased_2013,mcpherson_birds_2001,mas_differentiation_2013}
among other sources. They are also known to prefer interactions with those who
are similar to them in some respect (the well-known effect of
``homophily''\cite{boucher_structural_2015,centola_homophily_2007}) and prone
to sever social ties with those who believe
differently\cite{skoric_what_2018,paik_defriending_2023}.

Further, political parties and candidates are known to adjust their positions
on issues in a strategic attempt to appeal to greater numbers of
voters.\cite{worcester_voter_2006} This phenomenon, termed ``policy
positioning''\cite{mcelroy_policy_2012} or ``issue
adaptation''\cite{benoit_issue_2002} means that, like voters, candidates
dynamically vary their articulated political opinions over time. Neither voters
nor candidates occupy a fixed position in opinion space.

This dynamic interplay between voters influencing one another and candidates
seeking to gain their allegiance is worth studying, in part because the results
it will produce are unknown at the outset. Will candidates who ``chase'' voters
in this way succeed in winning more elections? Or will they alienate their base
by deserting policy positions they previously held, wiping out these gains? If
voters' opinions are ``drifting'' in response to one another between election
cycles, will candidates be chasing a moving target and thus outfox themselves?

We purport to answer such questions with an agent-based model that reproduces
this dance in opinion space between voters and candidates. Opinions on issues
are represented as vectors of values in a continuous n-dimensional space (as in
works like \cite{fortunato_vector_2005,lorenz_continuous_2007}), and voters
interact with one another on a randomly-generated social network, influencing
each other on these issues. Concurrently, candidates evaluate their estimated
vote totals and adjust their policy positions in an attempt to maximize them.
Voters are each equipped with one of a variety of empirically-verified voting
strategies, and cast their ballots according to that algorithm. With this
model, we can study the effects of different voting algorithms within the
electorate, and different strategic choices on the part of candidates, on
election results.



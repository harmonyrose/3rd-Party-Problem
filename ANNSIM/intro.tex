\section{Introduction}
\label{sec:intro}

Many factors are known to influence voters in democratic elections, including
their own background characteristics (demographics, ideology, partisanship,
personality traits, \textit{etc.}), the distinctiveness of the candidates, the
availability of information, and the voter's perception of their
identity.\cite[pp.3-4]{redlawsk_citizens_2020} But another major factor -- and
one we would argue \textit{should} be the most important, if citizens are
casting their votes rationally -- is the candidates' stances on the issues of
the day. At least in principle, voters are electing the candidate who will
enact policies most favorable to them if elected to office.

Voters are known to adjust their positions on issues over time, in response to
influence from their
peers\cite{dandekar_biased_2013,mcpherson_birds_2001,mas_differentiation_2013}
among other sources. They are also known to prefer interactions with those who
are similar to them in some respect (the well-known effect of
``homophily''\cite{boucher_structural_2015,centola_homophily_2007}) and
to sever social ties with those who believe
differently\cite{skoric_what_2018,paik_defriending_2023}.


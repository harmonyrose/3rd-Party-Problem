\section{The model}
\label{sec:model}

\subsection{Overview}

%1. Overall description: voters and candidates interact in opinion space **(SD)**

\subsection{Voter agents}

%2. Voter agents **(SD)**
%    1. Modeling opinions on issues as continuous-valued vectors
%    2. How voters choose who to interact with
%    3. What happens at interaction -- CI2
%         Use the terms "openness" and "pushaway"
%         Use the terms "comparison issue" and "influenced issue"
%    4. Turnout: an voter's decision about whether or not to vote

\subsection{Elections}

%5. Elections **(HP)**
%    1. At regular points in time, elections are held, and results tabulated

To study the political impact of the CI2 mechanism and explore the effects of issue-based
campaigning and voting, we introduce elections to the model. Elections are held at regular
intervals, every $N=50$ steps, and results are tabulated to determine the winners. 

\subsubsection{Voting algorithms}

%    5. Different voting algorithms **(HP)**
%        * Rational
%        * Bounded Rational (or "Constrained Rational")
%        * F&F1
%        * F&F2
%        * Party-line vote

At each election, all agents deploy one of several voting algorithms, modeling the
discrepancies in the thinking patterns and decision-making strategies of real voters.
Derived from the literature on voter psychology, the voting algorithms are as follows:
\begin{enumerate}
	\item Rational: Vote for the candidate whose opinions are closest to the voting agent's in
	Euclidean distance.
	\item Party-line: Vote for the candidate belonging to the same party as the voting agent.
	\item Fast \& Frugal \#1 (F\&F1):  
	\item Fast \& Frugal \#2 (F\&F2):
\end{enumerate}


\subsection{Parties}

%3. Parties **(HP)**
%    1. How these are initialized, both for voters and candidates
%    2. Whether and how voters ever change parties

\subsection{Candidate agents}

%4. Candidate agents **(HP)**
%    1. Their issue positions are modeled similar to voter opinions
%    2. The "chase" algorithm
%         Define the terms "chasing candidate" and "non-chasing candidate"


\section{The model}
\label{sec:model}

\subsection{Overview}

%1. Overall description: voters and candidates interact in opinion space **(SD)**

\subsection{Voter agents}

\subsubsection{Voting algorithms}

%2. Voter agents **(SD)**
%    1. Modeling opinions on issues as continuous-valued vectors
%    2. How voters choose who to interact with
%    3. What happens at interaction -- CI2
%    4. Turnout: an voter's decision about whether or not to vote
%    5. Different voting algorithms **(HP)**
%        * Rational
%        * Bounded Rational (or "Constrained Rational")
%        * F&F1
%        * F&F2
%        * Party-line vote

\subsection{Parties}

%3. Parties **(HP)**
%    1. How these are initialized, both for voters and candidates
%    2. Whether and how voters ever change parties

\subsection{Candidate agents}

%4. Candidate agents **(HP)**
%    1. Their issue positions are modeled similar to voter opinions
%    2. The "drift" algorithm

\subsection{Elections}

%5. Elections **(HP)**
%    1. At regular points in time, elections are held, and results tabulated

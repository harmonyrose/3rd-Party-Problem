\section{The model}
\label{sec:model}

\subsection{Overview}

An abstract \textbf{opinion space} is represented as a continuous
$I_N$-dimensional Euclidean space, where $I_N$ is the number of issues. An
\textbf{opinion} $O_i$ on \textbf{issue} $I_i$ is a value in the real interval
$[0,1]$, for $1 \leq i \leq I_N$.

There are two kinds of agents in the model: $C_N$ \textbf{candidates}, and
$V_N$ \textbf{voters}. Agents of either type each have a dynamic
\textbf{opinion vector} $(O_{i_1}, O_{i_2}, O_{i_3}, ..., O_{i_{I_N}})$
comprising one opinion on each issue; in other words, a point in the opinion
space.

A random Erd\H{o}s-R\'{e}enyi graph\cite{erdhos_evolution_1960} is generated
with $V_N$ nodes and uniform edge probability of $p_e$. During each iteration
of the simulation, voters interact randomly with their graph neighbors in
pairwise fashion, as explained below. This often results in one of the opinions
of one voter in the pair being moved either higher or lower in the interval.
We term this movement of voter opinions ``\textbf{drifting}.''

Every $E$ (``election interval'') iterations, each candidate adjusts (within
constraints) its opinion vector to the point in opinion space that would
produce the maximum number of votes assuming (1) all voter agents vote
rationally (\textit{i.e.} they will vote for the candidate whose opinion vector
is closest to theirs in opinion space) and (2) no other candidates' opinion
vector changes. We term this movement of candidate opinions
``\textbf{chasing}.''

\subsection{Voter agents}

At each iteration of the simulation, each agent $V_i$ (in randomly-chosen
order) has the opportunity to be influenced according to the
\textbf{cross-issue influence} (CI2) algorithm\cite{davies_agent-based_2023}.
To do so, it chooses one of its graph neighbors $V_j$ at random. It also
selects two of the $I_N$ issues, $I_c$ and $I_f$ (with $I_c \neq I_f$) as the
\textbf{comparison issue} and the \textbf{influenced issue}, respectively.

Agent $V_i$ then compares its opinion on issue $I_c$ ($O_{i_c}$) with agent
$V_j$'s opinion on that issue ($O_{j_c}$). If $|O_{i_c} - O_{j_c}| < T_o$,
where $T_o$ is the \textbf{openness threshold}, agent $V_i$ considers $V_j$ to
be homophilous and therefore trustworthy. It will then move its opinion on
$I_f$ ($O_{i_f}$) to be the average of its current value and agent $V_j$'s
value on $I_f$ ($O_{j_f}$). (For example, suppose issue 9 is chosen as the
comparison issue and issue 2 is chosen as the influenced issue. Then suppose
$O_{i_9} = .2, O_{j_9} = .4$, and $T_o = .15$. Since $|O_{i_9} - O_{j_9}| = .2
< .15$, $O_{i_2}$ will be influenced midway towards $O_{j_2}$, and settle at
.3.)

On the other hand, if $|V_i - V_j| > T_p$, where $T_p$ is the \textbf{pushaway
threshold}, agent $V_i$ considers $V_j$ to be so dissimilar that its opinion on
the influenced issue $T_f$ will be repelled. In this case, agent $V_i$ will
move its opinion on $I_f$ to be midway between its current position and that of
the pole (either 0 or 1). (For example, suppose issue 4 is chosen as the
comparison issue and issue 5 is chosen as the influenced issue. Then suppose
$O_{i_4} = .2, O_{j_4} = .9$, and $T_p = .6$. Since $|O_{i_4} - O_{j_4}| = .7
> .6$, $O_{i_5}$ will be pushed away from $O_{j_5}$, and settle at .1.)


\subsection{Elections}

%5. Elections **(HP)**
%    1. At regular points in time, elections are held, and results tabulated

To study the political impact of the CI2 mechanism and explore the effects of
issue-based campaigning and voting, we introduce elections to the model.
Elections are held every $N=50$ steps and results
are tabulated to determine the winner, as well as the rationality of the
outcome. An election is considered \textbf{rational} if the winning candidate
is the candidate that best represents the population in opinion space. This is
close to what Redlawsk and Habegger describe about an individual
``\textbf{voting correctly}, or matching their values to the candidate who
represents them most accurately,''\cite[p.8]{redlawsk_citizens_2020} (emphasis
original) except that we look at the outcome of the entire election. 

\subsubsection{Voting algorithms}

%    5. Different voting algorithms **(HP)**
%        * Rational
%        * Bounded Rational (or "Constrained Rational")
%        * F&F1
%        * F&F2
%        * Party-line vote

At each election, all agents deploy one of several voting algorithms, modeling the
discrepancies in the thinking patterns and decision-making strategies of real voters.
Derived from the literature on voter psychology, the voting algorithms are as follows: 
rational, party-line, fast \& frugal \#1 (F\&F1), and fast \& frugal \#2 (F\&F2).
\\\\
Rational agents vote for the candidate they are closest to in Euclidean space. They are an
ideal-type voter, representing someone with perfect information and the ability to maximize 
utility. Party-line agents vote for the candidate that belongs to the same party as them. They vote in a manner that reaffirms their identity as a party member, ignoring positions on issues. F\&F1 agents vote based solely on one "core opinion" that is randomly determined at initialization and remains their core issue through the entire simulation. F\&F2 agents vote based on one "hot topic"
issue that changes with each election cycle. All F\&F2 voters vote for the candidate closest to them
on the same issue. F\&F agents embody those who make a calculated trade-off between a good decision and an easy decision. They do not consider all information, only that which is most important to them, and cast their votes based on that limited information, seeing as it is "good enough".


\subsection{Parties}

%3. Parties **(HP)**
%    1. How these are initialized, both for voters and candidates
%    2. Whether and how voters ever change parties

We initialize parties to represent each of the candidates

\subsection{Candidate agents}

%4. Candidate agents **(HP)**
%    1. Their issue positions are modeled similar to voter opinions
%    2. The "chase" algorithm
%         Define the terms "chasing candidate" and "non-chasing candidate"


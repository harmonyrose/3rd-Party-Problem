\section{The model}
\label{sec:model}

\subsection{Overview}

%1. Overall description: voters and candidates interact in opinion space **(SD)**

\subsection{Voter agents}

%2. Voter agents **(SD)**
%    1. Modeling opinions on issues as continuous-valued vectors
%    2. How voters choose who to interact with
%    3. What happens at interaction -- CI2
%         Use the terms "openness" and "pushaway"
%         Use the terms "comparison issue" and "influenced issue"
%    4. Turnout: an voter's decision about whether or not to vote

\subsection{Elections}

%5. Elections **(HP)**
%    1. At regular points in time, elections are held, and results tabulated

To study the political impact of the CI2 mechanism and explore the effects of
issue-based campaigning and voting, we introduce elections to the model.
Elections are held every $N=50$ steps and results
are tabulated to determine the winner, as well as the rationality of the
outcome. An election is considered \textbf{rational} if the winning candidate
is the candidate that best represents the population in opinion space. This is
close to what Redlawsk and Habegger describe about an individual
``\textbf{voting correctly}, or matching their values to the candidate who
represents them most accurately,''\cite[p.8]{redlawsk_citizens_2020} (emphasis
original) except that we look at the outcome of the entire election. 

\subsubsection{Voting algorithms}

%    5. Different voting algorithms **(HP)**
%        * Rational
%        * Bounded Rational (or "Constrained Rational")
%        * F&F1
%        * F&F2
%        * Party-line vote

At each election, all agents deploy one of several voting algorithms, modeling the
discrepancies in the thinking patterns and decision-making strategies of real voters.
Derived from the literature on voter psychology, the voting algorithms are as follows: 
rational, party-line, fast \& frugal \#1 (F\&F1), and fast \& frugal \#2 (F\&F2).
\\\\
Rational agents vote for the candidate they are closest to in Euclidean space. They are an
ideal-type voter, representing someone with perfect information and the ability to maximize 
utility. Party-line agents vote for the candidate that belongs to the same party as them. They vote in a manner that reaffirms their identity as a party member, ignoring positions on issues. F\&F1 agents vote based solely on one "core opinion" that is randomly determined at initialization and remains their core issue through the entire simulation. F\&F2 agents vote based on one "hot topic"
issue that changes with each election cycle. All F\&F2 voters vote for the candidate closest to them
on the same issue. F\&F agents embody those who make a calculated trade-off between a good decision and an easy decision. They do not consider all information, only that which is most important to them, and cast their votes based on that limited information, seeing as it is "good enough".


\subsection{Parties}

%3. Parties **(HP)**
%    1. How these are initialized, both for voters and candidates
%    2. Whether and how voters ever change parties

We initialize parties to represent each of the candidates

\subsection{Candidate agents}

%4. Candidate agents **(HP)**
%    1. Their issue positions are modeled similar to voter opinions
%    2. The "chase" algorithm
%         Define the terms "chasing candidate" and "non-chasing candidate"


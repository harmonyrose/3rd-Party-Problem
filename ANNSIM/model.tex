\section{The model}
\label{sec:model}

\subsection{Overview}

%1. Overall description: voters and candidates interact in opinion space **(SD)**

\subsection{Voter agents}

%2. Voter agents **(SD)**
%    1. Modeling opinions on issues as continuous-valued vectors
%    2. How voters choose who to interact with
%    3. What happens at interaction -- CI2
%         Use the terms "openness" and "pushaway"
%         Use the terms "comparison issue" and "influenced issue"
%    4. Turnout: an voter's decision about whether or not to vote

\subsection{Elections}

%5. Elections **(HP)**
%    1. At regular points in time, elections are held, and results tabulated

To study the political impact of the CI2 mechanism, we introduce a simple political
structure in which all agents deploy one of several voting strategies in regularly held elections.
Elections are held every $N=50$ steps and results are tabulated to determine the winner, as well 
as the rationality of the outcome. An election is considered rational if the
winning candidate is the candidate that best represents the population
in opinion space.

\subsubsection{Voting algorithms}

%    5. Different voting algorithms **(HP)**
%        * Rational
%        * Bounded Rational (or "Constrained Rational")
%        * F&F1
%        * F&F2
%        * Party-line vote



\subsection{Parties}

%3. Parties **(HP)**
%    1. How these are initialized, both for voters and candidates
%    2. Whether and how voters ever change parties

\subsection{Candidate agents}

%4. Candidate agents **(HP)**
%    1. Their issue positions are modeled similar to voter opinions
%    2. The "chase" algorithm
%         Define the terms "chasing candidate" and "non-chasing candidate"

